\documentclass[handout]{beamer}
%\documentclass{beamer}
% \usetheme[faculty=fi]{fibeamer}
\usetheme{CambridgeUS}
%%\usepackage[T1]{fontenc}
\usepackage{fontspec}
%%\setmainfont{FreeSerif}
\usepackage{fontawesome}
\usepackage{polyglossia}
\setmainlanguage{czech}
\usepackage{lmodern}
\usepackage{xunicode}
\usepackage{xltxtra}
%%\usepackage[czech]{babel}
\usepackage{hyperref}
\usepackage{minted}
%\usemintedstyle{manni}
\title{POSIX, Make, CMake}
\author{Miroslav Jaroš}
\subject{Úvod do nízkoúrovňového programování}
\institute[PB071]{PB071 Úvod do nízkoúrovňového programovnání}

%%\newfontfamily{\FA}{FontAwesome}

\AtBeginSection[]{
    \begin{frame}
        \vfill
        \centering
        \begin{beamercolorbox}[sep=8pt,center,shadow=true,rounded=true]{title}
            \usebeamerfont{title}\insertsectionhead\par%
        \end{beamercolorbox}
        \vfill
    \end{frame}
}

\newmintinline{c}{}

\begin{document}

\frame{\titlepage}

\begin{frame}
    \frametitle{Obsah}
    \tableofcontents
\end{frame}

\section{POSIX}
\subsection{Proč POSIX}
\small
\begin{frame}{Motivace}
    \framesubtitle{Proč se zabývat operačním systémem?}
    \begin{columns}
        \begin{column}{0.74 \linewidth}
            \begin{itemize}
                \item<1-> Standardní knihovna přináší programátorovi základní funkce pro práci s prostředky počítače.
                \item<2-> Nicméně kdykoliv program potřebuje interagovat s hw musí požádat OS o zpřístupnění:
                    \begin{itemize}
                        \item<3-> Práce se soubory,
                        \item<4-> Požadavek na naalokování stránky do virtuální paměti,
                        \item<5-> Spuštění podprocesu v shellu,
                        \item<6-> Řízení vláken (od C11).
                    \end{itemize}
                \item<7-> Jak standardní knihovna zvládá toto všechno implementovat?
            \end{itemize}
        \end{column}\hfill
        \begin{column}{0.24\linewidth}<8->
            \includegraphics[width=1.0\textwidth]{pictures/square_bob.jpeg}
        \end{column}
    \end{columns}
\end{frame}
\begin{frame}{Motivace}
    \begin{center}
        \includegraphics[height=0.7\textheight]{pictures/grumpy.jpeg}
    \end{center}
\end{frame}
\begin{frame}{Historie}
    \framesubtitle{Aneb od UNIXu ke standardu}
    \begin{itemize}
        \item \texttt{POSIX -- Portable Operating System Interface}
        \item Norma pro rozhraní operačního systému založená na operačním systému \texttt{UNIX}.
            \begin{itemize}
                \item UNIX vznikl roku 1971 a již roku 1973 byl přepsán do jazyka C.
                \item Jeho autoři jsou Dennis M. Ritchie a Ken Thompson.
            \end{itemize}
        \item Součástí normy POSIX je knihovna pro jazyk C -- \texttt{POSIX C Library}, která definuje základní rozhraní operačního systému.
        \item Pro použivání ochrané známky UNIX musí operační systém plně implementovat normu POSIX a být certifikovaný podle Single Unix Specification.
            \begin{itemize}
                \item Např. macOS je certifikovaný UNIX.
                \item Linux není.
            \end{itemize}
        \item Nicméně většina \texttt{UNIX-like} (nebo také UN*X) systémů jej dodržuje (s~odchylkami).
    \end{itemize}
\end{frame}
\subsection{POSIX C Library}
\begin{frame}{POSIX C Library}
    \framesubtitle{API operačního systému}
    \begin{itemize}
        \item Zpřistupňuje funkce pro interakci s operačním systémem.
        \item Snaží se o vytvoření API kompatibilního se standardní knihovnou C, která je její součástí.
        \item Pokrývá širokou škálu služeb jádra poskytovaných programům:
            \begin{itemize}
                \item Správa procesů (start, komunikace, ukončení);
                \item Přístup k souborovému systému, síťovému rozhraní, roury;
                \item Správa a synchronizace vláken (spouštění, vyloučení přistupu, semafory \ldots);
                \item Správa virtuální paměti procesu (mapování stránek, dealokace \ldots);
                \item A mnoho dalších \ldots
            \end{itemize}
        \item Tím umožňuje psaní ``přenositelných'' programů s daleko širším záběrem, než má standardní knihovna.
    \end{itemize}
\end{frame}
\begin{frame}{Kompatibilita}
    \begin{itemize}
        \item \texttt{UN*X} systémy (Linux, macOS, Solaris \ldots)
            \begin{itemize}
                \item Pokud je systém certifikován jako UNIX, potom splňuje POSIX.
                \item Linux jej s minimálními odchylkami splňuje taktéž.
            \end{itemize}
        \item{Windows}
            \begin{itemize}
                \item Má vlastní API operačního systému \texttt{Win32} a \texttt{WinRT}.
                \item Nicméně části POSIX implementuje, ale ne plně.
                \item MinGW a Cygwin implementuji POSIX prostředí pomocí \texttt{Win32} API.
                \item Od Windows 10 obsahuje \texttt{Windows Subsystem for Linux}.
                    \begin{itemize}
                        \item Emuluje rozhraní Linuxu, pro běh linuxových aplikací.
                        \item Pro instalaci a další zdroje viz tutorial.
                        \item \url{https://www.fi.muni.cz/pb071/tutorials/ubuntu-on-windows/index.html}
                    \end{itemize}
            \end{itemize}
    \end{itemize}
\end{frame}
\subsection{Adresáře a soubory}
\begin{frame}[allowframebreaks,fragile]{Souborový systém}
    \begin{itemize}
        \item V UNIX-like systémech je souborový systém implementován jako n-ární strom s jedním kořenem.
        \item Všechny svazky (jiné disky, síťová úložište, ale i zařízení) jsou v něm adresovány - \texttt{mount}.
        \item Kořen souborového systému se jmenuje \texttt{/}.
        \item Soubory jsou implementovány pomocí \texttt{inode} (i-uzel).
        \begin{itemize}
            \item \texttt{inode} je datová struktura popisující objekt existující v souborovém systému.
                \item Váží se na něj všechny atributy souboru, jako velikost, oprávnění, datum vytvoření, modifikace atd.
                \item \texttt{inode} nicméně neobsahuje informaci o tom, jak se daný soubor jmenuje, nebo kde se ve FS nachází.
            \end{itemize}
        \item Vazba mezi jménem souboru a jeho \texttt{inode} je implementována na úrovni adresáře.
                \item Pro získání informací o souboru se používají funkce \texttt{stat, fstat, lstat}.
                \item \cinline[fontsize=\footnotesize]{ int stat(const char *path, struct stat *buf);}
                \item Při úspěchu je \cinline{struct stat} naplněna informacemi o souboru.
                \item Více viz \mintinline{bash}{man 3 stat}.
    \end{itemize}
%%    \framebreak
\begin{minted}[fontsize=\footnotesize]{c}
struct stat {
    dev_t     st_dev;         /* ID of device containing file */
    ino_t     st_ino;         /* Inode number */
    mode_t    st_mode;        /* File type and mode */
    nlink_t   st_nlink;       /* Number of hard links */
    uid_t     st_uid;         /* User ID of owner */
    gid_t     st_gid;         /* Group ID of owner */
    dev_t     st_rdev;        /* Device ID (if special file) */
    off_t     st_size;        /* Total size, in bytes */
    blksize_t st_blksize;     /* Block size for filesystem I/O */
    blkcnt_t  st_blocks;      /* Number of 512B blocks allocated */
}
\end{minted}
\end{frame}
\begin{frame}{Práce se soubory}
    \begin{itemize}
        \item Velice podobná jako ve standardní knihovně.
        \item Místo struktury \cinline{FILE *} se používá \texttt{file deskriptor}, který je typu \cinline{int}.
        \item \mintinline{c}{int open(const char *path, int oflag, ...);}
        \item \mintinline{c}{int close(int fd);}
        \item \mintinline{c}{ssize_t read(int fildes, void *buf, size_t nbyte);}
        \item \mintinline{c}{ssize_t write(int fildes, const void *buf, size_t nbyte);}
        \item \cinline{ssize_t} je rozšírení typu \cinline{size_t} o záporná čísla.
        \item Manuálové stránky těchto funkcí dle POSIX \mintinline{bash}{man 3 $JMENO_FUNKCE}.
    \end{itemize}
\end{frame}
\begin{frame}{Adresáře}
    \begin{itemize}
        \item Adresář je specifická entita v rámci souborových systému a jeho formát na souborovém systému záleží.
        \item Stejně jako u souborů (jimiž ve skutečnosti většinou i bývají) je práce s nimi
            programátorovi zpřístupněna operačním systémem.
        \item Pro práci s adresáři se používá kombinace funkcí \texttt{opendir}, \texttt{readdir} a \texttt{closedir}.
        \item \cinline{DIR *opendir(const char *name);}
            \begin{itemize}
                \item Vrací ukazatel na \texttt{DIR}, což je reprezentace otevřeného adresáře pro OS.
            \end{itemize}
        \item \cinline{struct dirent *readdir(DIR *dirp);}
            \begin{itemize}
                \item Přečte další položku v adresáři.
                \item Pokud v adresáři není žádný další prvek, vrací \texttt{NULL}.
            \end{itemize}
        \item \cinline{int closedir(DIR *dirp);}
            \begin{itemize}
                \item Ukončuje práci s adresářem a uvolňuje zdroje.
                \item I zde platí stejně jako u \texttt{fopen} nebo \texttt{malloc}, že pokud funkce \texttt{opendir} neselhala, musí být
                    nad její návratovou hodnotou zavolána funkce \texttt{closedir}.
            \end{itemize}
    \end{itemize}
\end{frame}
\begin{frame}[fragile]{Adresáře II.}
    Po zavolání funkce \texttt{readdir} je vrácena \texttt{struct dirent *}.
    \begin{minted}[fontsize=\tiny]{c}
struct dirent {
       ino_t          d_ino;       /* Inode number */
       off_t          d_off;       /* Not an offset; see below */
       unsigned short d_reclen;    /* Length of this record */
       unsigned char  d_type;      /* Type of file; not supported
                                      by all filesystem types */
       char           d_name[256]; /* Null-terminated filename */
};
    \end{minted}
    \begin{itemize}
        \item Položka \texttt{d\_name} obsahuje jméno prvku ve složce, ale ne cestu, ta musí být zrekonstruována jiným způsobem.
        \item Položka \texttt{d\_type} může obsahovat typ prvku, ale v závislosti na použitém souborovém systému také nemusí.
        \item Pro zjištění typu lze \texttt{d\_type} testovat proti konstantám:
            \begin{itemize}
                \item \texttt{DT\_REG} běžný soubor,
                \item \texttt{DT\_DIR} adresář,
                \item \texttt{DT\_UNKNOWN} File system nepodporuje \texttt{d\_type} a typ je potřeba zjistit jinak, například voláním \texttt{lstat},
                \item A dalším (viz \texttt{man 3 readdir}).
            \end{itemize}
    \end{itemize}
\end{frame}
\begin{frame}[fragile]{Procházení adresáře}
    \begin{listing}[H]
    \begin{minted}[fontsize=\footnotesize]{c}
void posix_print_files(const char *path) {
    DIR *dir = NULL;
    if ((dir = opendir(path)) != NULL) { // connect to directory
        struct dirent *dir_entry = NULL;
        while ((dir_entry = readdir(dir)) != NULL) {// obtain next item
            printf("File %s\n", dir_entry->d_name); // get name
        }
        closedir(dir); // finish work with directory
    }
}
    \end{minted}
    \caption{Procházení FS, převzato a upraveno ze starší přednášky Šimona Totha}
\end{listing}
\end{frame}
\subsection{Procesy a vlákna}
\begin{frame}{Procesy}
    \begin{itemize}
        \item Proces je v OS jednotka pro běh samostatného programu s vlastní oddělenou pamětí.
        \item V rámci POSIX má proces všechny zdroje (pokud nejsou sdílené) alokované pro vlastní použití (například file deskriptory).
        \item V rámcí standardní knihovny existuje funkce \texttt{system}, která spouští shell v novém procesu a blokuje rodičovský proces,
            dokud potomek neskončí.
            \begin{itemize}
                    \item V Linuxu je tato funkce implementována jako sekvence volání \texttt{fork(2), execl(3) a waitpid(3)}.
            \end{itemize}
        \item Spuštění nového procesu \cinline{pid_t fork(void);}
            \begin{itemize}
                \item Vytváří nový proces zkopírováním celé virtuální paměti rodičovského procesu.
                \item Oba procesy, jak rodič tak potomek, pokračují ve vykonávání instrukcí na následujícím řádku po volání fork.
                \item Zda je proces rodič nebo potomek, lze zjistit pomocí návratové hodnoty fork.
            \end{itemize}

    \end{itemize}
\end{frame}
\begin{frame}{Procesy II.}
    \begin{itemize}
        \item Rodina funkcí \texttt{exec(3)}
            \begin{itemize}
                \item Spouští předaný program nahrazením kódu běžícího procesu kódem programu předaného funkci.
                \item Volání pouze nahrazuje výkonný kód procesu, ale nemění zdroje alokované u OS (např. tabulka file descriptorů).
                \item \cinline[fontsize=\scriptsize]{int execve(const char *path, char *const argv[], char *const envp[]);}
            \end{itemize}
        \item Funkce \texttt{popen(3)}
            \begin{itemize}
                \item \cinline{FILE *popen(const char *command, const char *type);}
                \item Narozdíl od funkce system spouští proces asynchronně, tedy bez blokování rodiče.
                \item Návratová hodnota funkce je rourou pro komunikaci s procesem.
                \item Může být pouze pro čtení nebo zápis, nikoliv obojí.
                \item Po skončení práce s procesem musí být nad návratovou hodnotou funkce zavoláno \texttt{pclose}, nikoliv \texttt{fclose}.
            \end{itemize}
    \end{itemize}
\end{frame}
\begin{frame}{Vlákna}
    \begin{itemize}
        \item Vlákna umožňují procesu vykonávat několik paralelních činností zároveň.
        \item Zároveň ale všechna vlákna mezi sebou sdílí prostředky (například paměť).
        \item Což může přinášet problémy \faLongArrowRight $\,$ souběh, uváznutí a další.
        \item Podpora pro vlákna je implementována v knihovně \texttt{pthread}.
        \item Umožňuje vlákna spouštět, nebo plánovat jejich odstranění.
        \item Zároveň obsahuje techniky pro synchronizaci vláken:
            \begin{itemize}
                \item Mutexy,
                \item Conditional variable.
            \end{itemize}
        \item Více o vláknech se dozvíte v předmětech PB152 nebo PB153
    \end{itemize}
    Více o těchto funkcích například ve slidech PB065 \cite{kas}.
\end{frame}
%\begin{frame}{Souborový systém}
\section{Make, CMake}
\begin{frame}{Překládání projektů}
    \begin{itemize}
        \item Prozatím jste se v tomto předmětu setkali s jednoduchými prográmky s jednotkami
            překladových jednotek, které šlo přeložit jedním příkazem.
        \item \mintinline{bash}{gcc -std=c99 -Wall -Wextra -pedantic -Werror *.c}
        \item<2-> Tento přístup lze aplikovat na jednoduché programy, které se přeloží okamžitě.
        \item<3-> Co ale s projekty, které obsahují tisíce řádků zdrojového kódu ve stovkách souborů.
        \begin{itemize}
            \item<4-> I malá změna v jednom zdrojovém souboru znamená překompilování všech zdrojových souborů.
            \item<4-> Což pro velké projekty trvá opravdu dlouho (linux cca 20 minut, gcc cca 1.5 hodiny).
        \end{itemize}
        \item<5-> Jak reagovat na případy, kdy potřebujeme více spustitelných souborů?
            \begin{itemize}
                \item<6-> Systémový démon a jeho CLI.
                \item<6-> Klient server aplikace.
            \end{itemize}
        \item<6-> A co závislosti? Kde najdeme knihovny, jak se jmenuji?
    \end{itemize}
\end{frame}
\begin{frame}{Překládání projektů II - Přístupy}
    \begin{itemize}
        \item Přímý překlad není moudrý, je nepřenositelný a těžko reprodukovatelný.
        \item Trošku lepším řešením je napsat si nějakou vlastní automatizaci.
            \begin{itemize}
                \item Typicky ve forme skriptu.
                \item Tento přístup možná vyřeší problém překladu více binárek a opakovatelnosti sestavení, ale!
                \item Překládat pouze změněné překladové jednotky je poměrně těžké implementovat.
                \item Pro vyřešení vztahů mezi zdrojovými soubory, bychom museli implementovat alespoň základní parser
                    jazyka pro jejich pochopení.
                \item A nezapomínejme na chyby.
                \end{itemize}
            \item Vůbec nejlepším řešením je použít nástroje k tomu určené.
    \end{itemize}
\end{frame}
\subsection{make}
\begin{frame}[allowframebreaks]{make}
    \begin{itemize}
        \item Nástroj na automatizaci sestavení.
        \item Popis sestavení projektu se ukládá do \texttt{Makefile}.
        \item Využívá rekurzivních pravidel pro vystavění stromu závislostí, který následně projde a nad každým cílem
            vykoná požadovanou akci
        \item Základním kamenem syntaxe je pravidlo \mintinline{make}{target: source1 source2}
            \begin{itemize}
                \item \texttt{target} cíl, typicky se jedná o produkovaný soubor (binárka, nebo mezilehlá překladová jednotka), případně jde o \texttt{phony} cíl (viz dále).
                \item \texttt{source1 source2} zdrojové soubory, může jít o soubory vytvořené předchozím pravidlem (cíle jiného pravidla), nebo zdrojové soubory překládaného programu.
            \end{itemize}
        \item Akce se nad cílem vykoná, pokud: 
            \begin{itemize}
                \item cíl neexistuje a je potřeba jej vytvořit,
                \item libovolný ze zdrojových souborů byl upraven po vytvoření cíle (datum modifikace zdrojového souboru je vyšší než datum modifikace cíle),
                \item cíl je \texttt{phony}, tedy neprodukuje žádný soubor.
            \end{itemize}
        \item Spuštěním příkazu \texttt{make} se vyhledá soubor \texttt{Makefile} v lokálním adresáři.
        \item Pravidlo, které se začne vykonávat je první nalezené.
        \item Toto chování lze změnit explicitním zadáním očekávaného pravidla (\emph{např.} \texttt{make clean}).
        \item Proměnné makefile lze nastavit na jiné hodnoty při spuštění \mintinline{bash}{make all CC=clang}.
        \item Pro zrychlení překladu lze použít přepínač \texttt{-jN}, který překládá paralelně až \texttt{N} překladových jednotek.
        \item \mintinline{bash}{make -j5 all}
    \end{itemize}
    \begin{figure}
    \includegraphics[width=0.9\textwidth]{./pictures/dependencies.png}
    \caption{\footnotesize Popis závislostí v projektu, zdroj: Jiří Slabý, přednáška PB071 z roku 2017}
\end{figure}
\end{frame}
\begin{frame}[fragile]{Makefile}
    \begin{itemize}
        \item Základ syntaxe jsou příkazy \texttt{target: source1 source2}.
        \item Pod touto deklarací jsou tabulátorem odsazené příkazy, které se mají provést.
        \item Lze v něm deklarovat proměnné
            \begin{itemize}
                \item \texttt{VAR\_NAME=value}
                \item A následně je odkazovat jako \texttt{\$(VAR\_NAME)}
            \end{itemize}
        \item Zobecnění pravidel (v GNU Make):
            \begin{itemize}
                \item Speciální znak \texttt{\%} je zástupným symbolem.
                \item Lze jej použít pro konstrukci pravidel typu:
                \item \mintinline{makefile}{%.o: %c}, které říká, že libovolné pravidlo, končící na .o, závisí na stejně se 
                        jmenujícím pravidle končícím na .c
                    \item Ke každému pravidlu potom náleží další speciální proměnné
                    \begin{itemize}
                        \item \texttt{\$@} je jméno cíle.
                        \item \texttt{\$<} je první závislost.
                        \item \texttt{\$\^} jsou všechny závislosti
                    \end{itemize}
            \end{itemize}
    \end{itemize}
\end{frame}
\begin{frame}[fragile]{Makefile II}

    \begin{minted}{makefile}
    CC=gcc
    CFLAGS=-std=c99 -Wall -Wextra -pedantic -Werror

    all: myprog

    myprog: main.o test.o
        $(CC) $(CFLAGS) $^ -o $@

    %.o: %.c
        $(CC) $(CFLAGS) -c -o $@ $^
    
    clean: 
        rm -f *.o myprog

    .PHONY: all clean
    \end{minted}
\end{frame}
\subsection{Generátory}
\begin{frame}{Generátory}
    \begin{itemize}
        \item \texttt{make} je silný nástroj pro popis sestavení projektu, nicméně neřeší všechny problémy.
        \item Je nezbytné stále popisovat závislosti ručně. (Nebo si vypomáhat dodatečnými soubory třeba s pomocí \texttt{gcc -MM})
        \item Má komplikovanou syntaxi a pro velké projekty již nedostačuje.
        \item Je méně přenositelný a nemá přímou podporu pro ověření vlastností kompilátoru nebo existence knihoven.
        \item Proto se pro větší projekty používají pokročílejší nástroje, které umí \texttt{Makefile} vygenerovat.
    \end{itemize}
\end{frame}
\begin{frame}{Generátory II -- Autotools}
    \begin{itemize}
        \item Jeden z nejrozšířenějších nástrojů v UNIXovém světe.
        \item Skládá se ze tří nástrojů:
            \begin{itemize}
                 \item Autoconf -- vyhledávání externích závislostí,
                 \item Automake -- popis překladu (podadresáře, layout projektu),
                \item Libtool -- podpora pro tvorbu knihoven.
            \end{itemize}
        \item Výsledkem je několik servisních souborů a jeden spustitelný skript \texttt{configure}.
        \item Po jeho spuštění začne testování OS, zda a kde má potřebné závislosti.
        \item Při úspěchu je vygenerován \texttt{Makefile}.
    \end{itemize}
\end{frame}

\begin{frame}[fragile]{Generátory III -- CMake}
    \begin{itemize}
        \item Se \texttt{CMake} se všichni minimálně od vidění známe :-)
        \item Základem konfigurace je jeden soubor \texttt{CMakeLists.txt}, který obsahuje popis projektu.
        \item Podobně jako u \texttt{Autotools} jeho spuštění generuje (krom obrovské spousty servisních souborů) \texttt{Makefile}.
        \item Ale neomezuje se pouze na ně, umí vygenerovat i projektové soubory pro \texttt{VisualStudio}, nebo \texttt{Ninja} build system.
    \end{itemize}
    \begin{listing}[H]
    \begin{minted}{makefile}
    cmake_minimum_required(VERSION 3.6)
    project(hw04)
     
    set(SOURCE_FILES test.cpp) 
    add_executable(hw04 ${SOURCE_FILES})
    \end{minted}
    \caption{Ukázka CMakeLists.txt}
\end{listing}
\end{frame}

\section{Závěr}
\begin{frame}{Závěr}
    \begin{itemize}
        \item Nebojte se Operačních systémů ani jejich API.
        \item Dávají vám do rukou silné zbraně při programování.
        \item Používejte manuálové stránky.
        \item A hlavně: Nepiště si vlastní nástroje pro sestavování projektů!
    \end{itemize}
\end{frame}
\begin{frame}{Kam dál?}
    \begin{itemize}
        \item \texttt{PV004} UNIX
        \item \texttt{PV065} UNIX -- programování a správa systému I
        \item \texttt{PB173} Tematický vývoj aplikací v C/C++ (skupina zaměřená na POSIX)
    \end{itemize}
\end{frame}
\begin{frame}
  \frametitle<presentation>{Další zdroje}    
  \begin{thebibliography}{10}    
  \beamertemplatearticlebibitems
  \bibitem{kas}
    Jan Kasprzak.
    \newblock {\em PV165 Unix programování a správa systému I}.
    \newblock \url{https://www.fi.muni.cz/~kas/pv065/pv065.pdf}.
  \beamertemplatearticlebibitems
  \bibitem{Makefile}
  Makefile reference
  \newblock \url{https://www.gnu.org/software/make/manual/html_node/index.html}
  \beamertemplatearticlebibitems
  \bibitem{Jemand2000}
  CMake tutorial
      \newblock \url{https://cmake.org/cmake-tutorial/}
    \end{thebibliography}
\end{frame}
\begin{frame}
    \huge Děkuji za pozornost
\end{frame}
\end{document}

